\documentclass{article}
\pagestyle{empty}
\usepackage[margin=0.75in]{geometry}
\usepackage{fancyvrb}
\usepackage{amsmath}
\usepackage{multicol}

\begin{document}

\centerline{\Large CSCI 301, Lab \# 2}
\centerline{\large Winter 2017}
\begin{multicols}{2}

\begin{description}
\item[Due:] Your program, named {\tt lab02.rkt}, must be submitted to
  Canvas before midnight, Tuesday, January 24.

\item[Background:] An infinite {\em continued fraction} is an expression
of the form
\[
f = \cfrac{n_1}{d_1 + \cfrac{n_2}{d_2 + \cfrac{n_3}{d_3 + \ddots}}}
\]
Where $n_i$ and $d_i$ are functions of the index, $i$.
For example, if $n_1=1$ and $d_1=1$ for all $i$, then we get an
expression equal to the reciprocal of the golden ratio:
\[
{\phi^{-1}} = \cfrac{1}{1 + \cfrac{1}{1 + \cfrac{1}{1 + \ddots}}}
\]
One way to approximate an infinite continued fraction is to truncate
the expansion after a given number of terms, giving a 
{\em k-term finite continued fraction}, which has the form:

\[
f = \cfrac{n_1}{d_1 + \cfrac{n_2}{d_2 + \cfrac{n_3}{\ddots + \cfrac{n_k}{d_k}}}}
\]


\item[Programming:]\mbox{}


Suppose that {\tt n} and {\tt d} are procedures of one argument 
(the term index $i$) that return the $n_i$ and $d_i$ of the terms of
the continued fraction.  Define a procedure {\tt cont-frac} such that
evaluating {\tt (cont-frac n d k)} computes the value of the $k$-term
finite continued fraction.

Now test your continued fraction implementation with several
  famous continued fractions:
\begin{enumerate}
\item Find $\phi$, which is approximately 1.61803398875, but you don't
  have to find that many digits.

 The main function call for this could look like:
  \begin{verbatim}
(cont-frac (lambda (x) 1.0)
           (lambda (x) 1.0)
           100)
\end{verbatim}
  Then take the reciprocal and compare it to the golden ratio.

\item
In 1737 Leonhard Euler showed that $e-2$
was equal to the continued fraction where all the numerators, $n_i$ were 1
and the denominators, $d_i$ were $1,2,1,1,4,1,1,6,1,1,8,\ldots$
Check your continued fraction procedure by using it to approximate $e$.  
You can find an accurate value of $e$ in
Scheme with {\tt (exp 1)}.

The main function call for this could look like this,
  where {\tt euler-d} is a function you'll have to write:
  \begin{verbatim}
(cont-frac (lambda (x) 1.0)
           euler-d
           100)
\end{verbatim}
  Then add 2 and compare it to $e$.


\item
J.H.Lambert showed in 1770 that the tangent function could be computed
with the continued fraction
\[
\tan(x) = \cfrac{x}{1 - \cfrac{x^2}{3 - \cfrac{x^2}{5 - \cfrac{x^2}{7
        - \ddots}}}}
\]
Observe the minus signs, and that the first $x$ is not squared.

Use your continued fraction procedure to 
write your own approximate tangent procedure, {\tt mytan}, and compare
your approximations for several angles with Scheme's builtin {\tt tan} 
function.

 The main function call for this should look like:
  \begin{verbatim}
(cont-frac (make-lambert-n x)   
           lambert-d
           100)
\end{verbatim}
  where {\tt lambert-d} and {\tt make-lambert-n} are functions you'll
  have to write.  {\tt make-lambert-n} is, of course, a function of
  a number, $x$, that returns a function of the index, $i$.  The
  returned function, in turn, will
  return $x$ if $i=1$ and $-x^2$ otherwise.

  Compare the result to $\tan(x)$, for several values of $x$.

\end{enumerate}


\end{description}


\end{multicols}
\end{document}


