\documentclass{article}
\pagestyle{empty}
%\usepackage[margin=0.75in]{geometry}
\usepackage{fancyvrb}
\usepackage{amsmath}

\begin{document}

\centerline{\Large CSCI 301, Lab \# 3}
\centerline{\large Winter 2017}


\begin{description}
\item[Due:] Your program, named {\tt lab03.rkt}, must be submitted to
  Canvas before midnight, Tuesday, Jan 31.

\item[Background:]  It's pretty easy to write a procedure to return
  the shorter of two lists using only {\tt car}, {\tt cdr}
  and {\tt null?}, if we first compute the length of each.  
For example, we can use:
\begin{Verbatim}[frame=single]
(define length 
  (lambda (ls) 
    (if (null? ls) 
        0 
        (+ 1 (length (cdr ls))))))

(define shorter
  (lambda (ls1 ls2)
    (if (<= (length ls1) (length ls2)) 
        ls1 
        ls2)))
\end{Verbatim}
Note: {\tt length} is actually builtin to DrRacket, but we show a
definition here to make a point.

However, this is very inefficient if the lists are vastly different in
length.  For example, if we had one list with length over a billion,
and the other with only 5 things in it, this procedure would traverse
{\em both} lists to the end, only to return the second list.  Clearly
we could have found out which was shorter by only examining the first
few of the items in each list.

\item[Programming:] Rewrite the {\tt shorter} procedure, so that it
  returns the same value as the procedure above, but does not use {\tt
    length}, and does not traverse both of the lists to the end when
  one is longer than the other.  You can write as many procedures as
  you like, but make sure:
\begin{itemize}
\item you use only {\tt car}, {\tt cdr} and {\tt null?} as builtin
  procedures,
\item your procedure
returns the shorter input list (or the first if they are the
  same length),
\item your procedure does not copy any lists,
\item traverses the minimum
  number of items to determine which list is shorter, and
\item 
  uses functional style, no assignment statements!
\end{itemize}

\end{description}


\end{document}

\end
