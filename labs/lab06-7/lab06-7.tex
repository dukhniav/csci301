\documentclass{article}
\usepackage[margin=0.75in]{geometry}
\usepackage{multicol}
\usepackage{fancyvrb}
\usepackage{hyperref}

\begin{document}

\centerline{\Large CSCI 301, Labs \# 6 and \# 7}
\centerline{\large Winter, 2017}

\begin{description}
\item[Due dates:]  \begin{itemize}\item
Lab \#6 must be submitted to Canvas before midnight
  Tuesday, February 28.
\item Lab \#7 must be submitted to Canvas before midnight
  Tuesday, March 7.
\end{itemize}

\item[Overall Problem:]
These labs are taken from an exercise found in another
online scheme book: \url{http://www.eecs.berkeley.edu/~bh/ss-toc2.html}.
Both of them together will create a 
procedure {\tt number-name} that takes a positive integer argument
and returns a sentence containing that number spelled out in words.
You do not have to worry about negative numbers, nor about the ``and''
that some people throw in before the last bunch.
\begin{Verbatim}[frame=single]
> (number-name 5513345)
(five million five hundred thirteen thousand three hundred forty five)
> (number-name (factorial 20))
(two quintillion four hundred thirty two quadrillion nine hundred two
 trillion eight billion one hundred seventy six million six hundred
 forty thousand)
> (number-name 0)
'(zero)
\end{Verbatim}
Note that the responses are lists of symbols, not strings.  You may
want to examine the difference in the documentation:
{\url{file:///usr/share/doc/racket/reference/symbols.html}}

Once again, {\bf no string processing!}


\item[Lab \#6:]  For the first part of this problem, 
  write an intermediate version of the solution, that behaves like this:
\begin{Verbatim}[frame=single]
> (number-name 1234567890)
'(1 billion 234 million 567 thousand 890)
> (number-name 1000000)
'(1 million)
> (number-name 1000000001)
'(1 billion 1)
> (number-name 2423000000000000002344)
'(2 sextillion 423 quintillion 2 thousand 344)
> (number-name 0)
'()
\end{Verbatim}
You will have to write a procedure to break up a big number into its
three-digit pieces (use {\tt quotient} and {\tt remainder}).
Note that if any slot is a zero, both it and its name are omitted.

\item[Lab \#7:]  
\begin{multicols}{2} This lab has two parts.
First, write a procedure {\tt name<1000} that converts a
  number less than 1000 into words, as shown at right.
Note that {\tt zero} only shows up if the whole number is zero. 

Second, put this together with the solution to lab \#4, and you should
solve the whole problem, with behavior as in the first box, above.

As a final wrinkle, note that {\tt zero} only shows up if the entire
number is zero.  We never say, for example, ``one million zero
thousands two,'' we just say ``one million two.''  

\vfill

\columnbreak
\begin{Verbatim}[frame=single]
> (name<1000 100)
'(one hundred)
> (name<1000 101)
'(one hundred one)
> (name<1000 15)
'(fifteen)
> (name<1000 111)
'(one hundred eleven)
> (name<1000 222)
'(two hundred twenty two)
\end{Verbatim}
\end{multicols}
\end{description}

\end{document}
