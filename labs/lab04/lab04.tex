\documentclass{article}
\pagestyle{empty}
%\usepackage[margin=0.75in]{geometry}
\usepackage{fancyvrb}
\usepackage{amsmath}

\begin{document}

\centerline{\Large CSCI 301, Lab \# 4}
\centerline{\large `Winter, 2017}


\begin{description}
\item[Due:] Your program, named {\tt lab04.rkt}, must be submitted to
  Canvas before midnight, Tuesday, February 7.

\item[Composition:]  Let $f$ and $g$ be two one-argument functions.
  The {\em composition} $f$ after $g$, written $f\circ g$, is 
  defined as $(f\circ g)(x) = f(g(x))$.
\begin{description}
\item[Programming composition:]
Define a procedure {\tt compose} that implements composition.  For
example:
\begin{Verbatim}
((compose square add1) 5) => 36
((compose add1 square) 5) => 26
\end{Verbatim}
\end{description}

\item[Repeated application:]  If $f$ is a numerical function and $n$ is
a non-negative integer, then we can form the $n$th repeated
application of $f$, which is defined to be the function whose value at
$x$ is $f(f(\ldots(f(x))\ldots))$, where $f$ is repeated $n$ times.
There seems to be no standard notation for this, but $f^{\circ n}(x)$
is popular.

For example, if $f$ is the function $f(x) = x+1$ then the $n$th
repeated application of $f$ is the function $f^{\circ n}(x)= x+n$.  If $f$ is
the operation of squaring a number, then the $n$th repeated
application of $f$ is the function that raises its argument to the
$2^n$th power.  
\begin{description}
\item[Programming repeated application:] 
Using {\tt compose} from the previous problem,
write a procedure that takes as inputs a procedure
that computes $f$ and a positive integer $n$ and returns the procedure
that computes the $n$th repeated application of $f$.  Your procedure
should be able to be used as follows:
\begin{Verbatim}
((repeated square 2) 5)  => 625
((repeated square 3) 5) => 390625
((repeated add1 50) 50) => 100
\end{Verbatim}

\item[Recursive and iterative:]
Write both a recursive and an iterative version of the {\tt repeated}
procedure.  The iterative version should use a tail-recursive function.

\end{description}
\end{description}


\end{document}

\end
