\documentclass{beamer}
\usepackage{fancyvrb}
\usepackage{hyperref}
\usepackage{multicol}
\usepackage{graphicx}
\newtheorem{theo}{Theorem}[section]

\newcommand{\myfig}[1]{\centerline{\includegraphics[scale=0.25]{figures/#1.png}}}

\newcommand{\trans}[5]{
\begin{tabular}{|c|c|c|c|c|}\hline
#1 & #2 & #3 & #4 & #5 \\\hline
\end{tabular}
}

\newcommand{\arr}{&\rightarrow&}
\newcommand{\darr}{&\Rightarrow&}
\newcommand{\ar}{\rightarrow}
\newcommand{\dar}{\Rightarrow}
\newcommand{\bee}{\begin{eqnarray*}}
\newcommand{\eee}{\end{eqnarray*}}
\newcommand{\lmb}{\ensuremath{\lambda}}

\newcommand{\bi}{\begin{itemize}}
\newcommand{\ii}{\item}
\newcommand{\ei}{\end{itemize}}

\newcommand{\sect}[1]{
\section{#1}
\begin{frame}[fragile]\frametitle{#1}
}

\mode<presentation>
{
%  \usetheme{Madrid}
  % or ...

%  \setbeamercovered{transparent}
  % or whatever (possibly just delete it)
}

\usepackage[english]{babel}

\usepackage[latin1]{inputenc}

\title
{
Scheme Notes 01
}

\subtitle{
} % (optional)

\author[Geoffrey Matthews]
{Geoffrey Matthews}
% - Use the \inst{?} command only if the authors have different
%   affiliation.

\institute[WWU/CS]
{
  Department of Computer Science\\
  Western Washington University
}
% - Use the \inst command only if there are several affiliations.
% - Keep it simple, no one is interested in your street address.

\date{\today}

% If you have a file called "university-logo-filename.xxx", where xxx
% is a graphic format that can be processed by latex or pdflatex,
% resp., then you can add a logo as follows:

%\pgfdeclareimage[height=0.5cm]{university-logo}{WWULogoProColor}
%\logo{\pgfuseimage{university-logo}}

% If you wish to uncover everything in a step-wise fashion, uncomment
% the following command: 

%\beamerdefaultoverlayspecification{<+->}

\begin{document}

\begin{frame}
  \titlepage
\end{frame}


\newcommand{\myref}[1]{\small\item\url{#1}}
\newcommand{\myreft}[1]{\footnotesize\item\url{#1}}

%\begin{frame}
%  \frametitle{Outline}
%  \tableofcontents
%  % You might wish to add the option [pausesections]
%\end{frame}

\sect{Resources}
\bi
\ii The software:
\bi\ii \url{https://racket-lang.org/}\ei
\ii Texts:
\bi
\ii \url{https://mitpress.mit.edu/sicp/}
\ii \url{http://www.scheme.com/tspl3/}\\ (make sure you use the 3rd
edition and not the 4th)
\ii \url{http://ds26gte.github.io/tyscheme/}
\ei
\ei
\end{frame}


\sect{Running the textbook examples}

\bi
\ii Using the racket language is usually best, the examples from
    {\em The Scheme Programming Language} should run without modification.

 \ii The examples from SICP are a little more idiosyncratic.  Most of
 them can be run by installing the {\tt sicp} package as in
 these instructions:\\
 \url{http://stackoverflow.com/questions/19546115/which-lang-packet-is-proper-for-sicp-in-dr-racket}

 \ei
\end{frame}

\sect{Every powerful language has}


\bi
\ii primitive expressions: the simplest entities, such as \verb|3| and \verb|+|
\ii
means of combination: buidling compound elements from simpler ones such as 

\verb|(+ 3 4)|
\bi
\ii In Scheme combinations are always parentheses, with the operator
first and the operands following.
\ei
\ii
means of abstraction: a way for naming compound elements and then manipulating them as units such as 

\verb|(define pi 3.14159)|

\verb|(define square (lambda (x) (* x x)))|
\ei

\end{frame}

\sect{The REPL does the following three things:}

\bi
\ii Reads an expression
\ii Evaluates it to produce a value
\ii Prints the value
\ei

The returned value has a small set of types, including number, boolean and procedure. (Later, we'll see symbol, pair, vector, and promise (stream).)

\end{frame}



\sect{There are 4 types of expressions:}
\bi
\ii Constants: numbers, booleans. Examples: \verb|4 3.141592 #t #f|
\ii Variables: names for values. We create these using the special form \verb|define|
\ii Special forms: have special rules for evaluation. In addition, you may not redefine a special form.
\ii Combinations: \verb|(<operator> <operands>)|. These are sometimes called "function calls" or "procedure applications."
\ei 

The first two types of expressions (constants and variables) are
primitive expressions -- they have no parentheses. The second two
types are called compound expressions -- they have parentheses.

\end{frame}

\sect{Mantras}
\bi
\ii Every expression has a value
\bi\ii(except for errors, infinite loops and the 
\verb|define| special form)\ei
\ii To find the value of a combination:
\bi
\ii Find values of all subexpressions in any order
\ii Apply the value of the first to the values of the rest
\ei
\ii The value of a \verb|lambda| expression is a procedure
\ei
\end{frame}

\sect{Finding the value of a combination}
\bi
\ii Find values of all subexpressions in any order
\ii Apply the value of the first to the values of the rest
\begin{verbatim}
    (+ (* 2 3) (- 8 2))

    (+ (* 2 3)    6   )

    (+    6       6   )

            12
\end{verbatim}
\ei
\end{frame}


\sect{Lambda}

When you hear the words "write a procedure," you should think of
\verb|lambda|.  Lambda is a special form that creates a
procedure. 

There are three parts to the lambda expression: 
\bi
\ii lambda
\ii parameter list 
\ii body
\ei
For example, let's write a procedure to square a
number: \verb|(lambda (x) (* x x))|
\bi
\ii
(x) is the parameter list. In this case, we only have one parameter.
\ii
(* x x) is the body of the lambda. The body of the lambda will not be
evaluated until the procedure is applied.
\ei
\end{frame}

\sect{Applying procedures}
\bi
\ii How do we use a procedure?
\ii We {\em apply} the procedure to some
arguments.
\ii Application consists of combining the procedure and its arguments
with parentheses.
\ii For example:

\verb|((lambda (x) (* x x)) 3)|

This will return 9.
\ei
\end{frame}


\sect{Abstraction}
\bi
\ii If we want to square 9, we can write
\verb|((lambda (x) (* x x)) 9)|
\ii
But we don't want to have to keep typing the procedure over and over.
\ii This leads us to another special form: \verb|define|
\ei
\end{frame}
\sect{Abstraction}

Define is a special form that allows us to name objects.
Define has three parts:
\bi
\ii {\tt define}
\ii name
\ii the object you want the name to be bound to
\ei
Here are some examples:
\bi
\ii\verb|(define pi 3.141592)|
\ii\verb|(define four 8)|
\ii\verb|(define square (lambda (x) (* x x)))|
\ei
Giving names to things is called {\em abstraction.}

\end{frame}

\sect{Abstraction}

\bi
\ii
We can use define to name our procedure we wrote above to allow us to use it without having to retype the lambda expression over and over.

\verb|(define square (lambda (x) (* x x)))|

\ii Now we can write

\verb|(square 9)|

\ii Instead of

\verb|((lambda (x) (* x x)) 9)|
\ei
\end{frame}


\sect{Special forms}

Special forms are those expressions that begin with an open
parenthesis followed by one of the 15 "magic words":

\begin{multicols}{4}
{\tt
and\\begin\\case\\cond\\define\\do\\if\\lambda\\let\\let*\\letrec\\or\\
quasiquote\\quote\\set!
}
\end{multicols}

\end{frame}

\sect{Special Forms Have Special Rules}
\bi
\ii
 Recall from the
mantras that to find the value of a combination, you find the values
of all of the subexpressions in any order.
\ii With special forms, this is
not done. 
\ii The order of the evaluation of the subexpressions is
specified for each special form.
\ei
\end{frame}

\sect{\tt and}

\verb|(and <exp1> <exp2> ...)|
\bi
\ii
\verb|and| evaluates the expressions one at a time in left to right
order. 
\ii As
soon as one of the expressions evaluates to \verb|#f| (false), the
value of the and expression is \verb|#f| (false) and none of the
remaining expressions are evaluated.
\ii If none of the expressions evaluates to \verb|#f| (false),
\verb|#t| (true) is
returned.
\ei
\end{frame}

\sect{\tt or}

\verb|(or <exp1> <exp2> ...)|
\bi
\ii
\verb|or| evaluates the expressions one at a time in left to right
order. 
\ii As soon as one of the expressions evaluates to
anything other than  \verb|#f| (false),
the value of the or expression is returned and none of the
remaining expressions are evaluated.
\ii If none of the expressions evaluates to something other than
\verb|#f|, then \verb|#f| (false) is
returned.
\ei
\end{frame}

\sect{\tt if}

\verb|(if <predicate> <consequent> <alternative>)|

\bi
\ii
The predicate is evaluated.
\ii If it is anything other than \verb|#f|, 
the value of the consequent will be returned.
\ii If it is \verb|#f|, the value of the alternative will be returned.
\ii In this special form, never will the consequent and alternative
both be evaluated. 
\ei
\end{frame}

\sect{\tt cond}
\begin{verbatim}
(cond (<pred1> <exp1>)
      (<pred2> <exp2>)
      ...
      (else <expn>))
\end{verbatim}

\bi
\ii
The first predicate is evaluated. 
\ii If it is anything other than \verb|#f|, 
the value of the first expression will be returned.
\ii If it is \verb|#f|, the second predicate will be evaluated.
\ii The computer will continue to evaluate the predicates until one is
something other than \verb|#f|.
\ii The value of the expression corresponding to the non-\verb|#f| predicate
will be returned.
\ii Note that the \verb|else| will always be true ({\em i.e.} not \verb|#f|).
\ei


\end{frame}





\end{document}


