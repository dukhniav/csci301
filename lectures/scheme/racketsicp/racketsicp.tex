\documentclass{article}
\title{SICP in DrRacket}
\author{Geoffrey Matthews}
\begin{document}
\maketitle

\begin{itemize}
\item
Most code from the SICP book can be run in DrRacket by using the R5RS
language.  The easiest way to use this language is to use the Racket
Language (the default in the ``Choose Language'' panel), and insert
the line:
\begin{verbatim}
#lang r5rs
\end{verbatim}
at the beginning of each file.  This will work except for the picture
language and examples that involve redefining initial bindings.

\item
You can also run the R5RS language by choosing ``Other Languages'' from
the ``Choose Language'' panel and clicking R5RS.  Doing this allows
you to go to the ``Show Details'' panel and uncheck the ``Disallow
redefinition of initial bindings'' checkbox, so you can now redefine
symbols as in some of the examples. Note that if you do this you
{\em cannot} have a first line starting with \verb|#lang|.  Remove it,
or comment it out.
\item
For the picture language, select the Racket language from the ``Choose
Language'' panel
with the following first line:
\begin{verbatim}
#lang planet neil/sicp
\end{verbatim}
The first time you
run this it will take a while and you will get many errors in RED.
These errors (and the long loading time) go away if you ignore them.

This will allow you to do the picture examples, such as:
\begin{verbatim}
(paint-hires  (below (beside diagonal-shading
                             (rotate90 diagonal-shading))
                     (beside (rotate270 diagonal-shading)
                             (rotate180 diagonal-shading))))
\end{verbatim}
\end{itemize}

\end{document}