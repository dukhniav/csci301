\documentclass{beamer}
\usepackage{fancyvrb}
\usepackage{hyperref}
\usepackage{multicol}
\usepackage{graphicx}
\newtheorem{theo}{Theorem}[section]

\newcommand{\myfig}[1]{\centerline{\includegraphics[scale=0.25]{figures/#1.png}}}

\newcommand{\trans}[5]{
\begin{tabular}{|c|c|c|c|c|}\hline
#1 & #2 & #3 & #4 & #5 \\\hline
\end{tabular}
}

\newcommand{\arr}{&\rightarrow&}
\newcommand{\darr}{&\Rightarrow&}
\newcommand{\ar}{\rightarrow}
\newcommand{\dar}{\Rightarrow}
\newcommand{\bee}{\begin{eqnarray*}}
\newcommand{\eee}{\end{eqnarray*}}
\newcommand{\lmb}{\ensuremath{\lambda}}

\newcommand{\bi}{\begin{itemize}}
\newcommand{\ii}{\item}
\newcommand{\ei}{\end{itemize}}

\newcommand{\sect}[1]{
\section{#1}
\begin{frame}[fragile]\frametitle{#1}
}

\mode<presentation>
{
%  \usetheme{Madrid}
  % or ...

%  \setbeamercovered{transparent}
  % or whatever (possibly just delete it)
}

\usepackage[english]{babel}

\usepackage[latin1]{inputenc}

\title
{
SICP Notes 02
}

\subtitle{
\url{https://mitpress.mit.edu/sicp/}\\
\url{http://www.aduni.org/courses/sicp/index.php}
} % (optional)

\author[Geoffrey Matthews]
{Geoffrey Matthews}
% - Use the \inst{?} command only if the authors have different
%   affiliation.

\institute[WWU/CS]
{
  Department of Computer Science\\
  Western Washington University
}
% - Use the \inst command only if there are several affiliations.
% - Keep it simple, no one is interested in your street address.

\date{\today}

% If you have a file called "university-logo-filename.xxx", where xxx
% is a graphic format that can be processed by latex or pdflatex,
% resp., then you can add a logo as follows:

%\pgfdeclareimage[height=0.5cm]{university-logo}{WWULogoProColor}
%\logo{\pgfuseimage{university-logo}}

% If you wish to uncover everything in a step-wise fashion, uncomment
% the following command: 

%\beamerdefaultoverlayspecification{<+->}

\begin{document}

\begin{frame}
  \titlepage
\end{frame}


\newcommand{\myref}[1]{\small\item\url{#1}}
\newcommand{\myreft}[1]{\footnotesize\item\url{#1}}

%\begin{frame}
%  \frametitle{Outline}
%  \tableofcontents
%  % You might wish to add the option [pausesections]
%\end{frame}

\sect{Mantra Review}

\bi
\ii
Every expression has a value (exceptions: errors, infinite loops and define)
\ii
To find the value of a combination,
\bi
\ii
Find the values of all of the subexpressions, in any order
\ii
Apply the value of the first to the values of the rest
\ei
\ii
The value of a lambda expression is a procedure
\ei
\end{frame}

\sect{Values of combinations}
\bi
\ii
To find the value of a combination,
\bi
\ii
Find the values of all of the subexpressions, in any order
\ii
Apply the value of the first to the values of the rest
\ei
\ei
\begin{Verbatim}[commandchars=\\\{\}]
(+ 2 3)\pause
(+ \fbox{2} 3)\pause
(+ \fbox{2} \fbox{3})\pause
(\fbox{+} \fbox{2} \fbox{3})\pause
5
\end{Verbatim}
\end{frame}




\sect{Values of combinations}
\bi
\ii
To find the value of a combination,
\bi
\ii
Find the values of all of the subexpressions, in any order
\ii
Apply the value of the first to the values of the rest
\ei
\ei
\begin{Verbatim}[commandchars=\\\{\}]
(+ 3 (- 8 2) (+ 2 3))\pause
(+ \fbox{3} (- 8 2) (+ 2 3))\pause
(+ \fbox{3} (- 8 2) \fbox{(+ 2 3)})\pause
(+ \fbox{3} (- 8 2) \fbox{5})\pause
(\fbox{+} \fbox{3} (- 8 2) \fbox{5})\pause
(\fbox{+} \fbox{3} \fbox{(- 8 2)} \fbox{5})\pause
(\fbox{+} \fbox{3} \fbox{6} \fbox{5})\pause
14
\end{Verbatim}

\end{frame}


\sect{Compound procedures}

\bi
\ii\verb|(lambda (x) (* x x))|
\ii
What is/are the parameters? \pause
\ii
\verb|x| is the only parameter in the parameter list \verb|(x)|\pause
\ii
What is the body?\pause
\ii
\verb|(* x x)| is the body of the lambda expression.
\ei
\end{frame}


\sect{Substitution Model}

To apply a compound procedure to arguments, evaluate the body of the
procedure with each formal parameter replaced by the corresponding
argument.

\begin{verbatim}
(define absolute-value (lambda (n) (if (< n 0) (- n) n)
\end{verbatim}


\begin{Verbatim}[commandchars=\\\{\}]
(absolute-value (+ 3 -8))\pause
(absolute-value \fbox{(+ 3 -8)})\pause
(absolute-value \fbox{-5})\pause
(\fbox{absolute-value} \fbox{-5})\pause
(\fbox{(lambda (n) (if (< n 0) (- n) n))} \fbox{-5})\pause
(if (< \fbox{-5} 0) (- \fbox{-5}) \fbox{-5}))\pause
(if \fbox{(< \fbox{-5} 0)} (- \fbox{-5}) \fbox{-5}))\pause
(if \fbox{#t} (- \fbox{-5}) \fbox{-5}))\pause
(- \fbox{-5})\pause
5
\end{Verbatim}
\end{frame}


\sect{Computing the Euclidean distance between two points}
\begin{Verbatim}
(define square
   (lambda (x) (* x x)))

(define sum-squares
   (lambda (x y) (+ (square x) (square y))))

(define dist-between-pts
   (lambda (x1 y1 x2 y2)
      (sqrt (sum-squares (- x1 x2) (- y1 y2)))))
\end{Verbatim}
Use the substitution model to evaluate (dist-between-pts 1 1 4 5):
\end{frame}

\sect{Applicative and Normal Order}

\bi \ii Applicative Order: Evaluate the arguments and then apply the
value of the first to the value of the rest. 
\bi\ii Everything is evaluated,
whether or not we use it. \ii This is the method Scheme uses and is the
reason that we need special forms. \ei
\ii Normal Order: Fully expand, then
reduce. Don't evaluate operands until they are needed. 
\ei
\end{frame}
\sect{Applicative and normal order}
\begin{Verbatim}
(define square
   (lambda (x) (* x x)))

(define sum-squares
   (lambda (x y) (+ (square x) (square y))))

(define dist-between-pts
   (lambda (x1 y1 x2 y2)
      (sqrt (sum-squares (- x1 x2) (- y1 y2)))))

(sum-squares (+ 5 1) (* 6 2)) 
\end{Verbatim}
\pause
\bi
\ii
Normal order is not as efficient as applicative order. We needed to
evaluate (+ 5 1) and (* 6 2) twice each using normal order, instead of
once each with applicative order. \pause
\ii How can we test if Scheme is
applicative or normal order?
\ei
\end{frame}


\sect{Applicative and normal order}

\verb|(define p (lambda () (p)))|

\bi
\ii What does this function do?\pause
\ii It calls itself repeatedly, causing an infinite loop.\pause
\ei

\begin{Verbatim}
(define test 
   (lambda (x y)
      (if (= x 0) 
          0
          y)))

(test 0 (p))
\end{Verbatim}
\bi
\ii What happens with applicative order?\pause
\ii We get an infinite loop.\pause
\ii What happens with normal order?\pause
\ii It returns 0.
\ei
\end{frame}

\sect{Writing our own {\tt if}}

What if \verb|if| were not a special form?
\begin{Verbatim}
(define new-if
   (lambda (predicate consequent alternative)
      (cond (predicate consequent)
            (else alternative))))
\end{Verbatim}
What happens when we evaluate \verb|(new-if (> 3 2) 0 2)|?
\begin{Verbatim}[commandchars=\\\{\}]
(new-if (> 3 2) 0 2)
(new-if (> 3 2) \fbox{0} 2)
(new-if (> 3 2) \fbox{0} \fbox{2})
(new-if \fbox{(> 3 2)} \fbox{0} \fbox{2})
(new-if \fbox{#t} \fbox{0} \fbox{2})
(cond (\fbox{#t} \fbox{0}) (else \fbox{2}))
0
\end{Verbatim}
No real problems so far. 
\end{frame}

\sect{Writing our own \tt if}

\begin{Verbatim}
(define fact
   (lambda (n)
      (if (= n 0)
          1
          (* n (fact (- n 1))))))
\end{Verbatim}
\bi
\ii
What if we use new-if instead of the special form if?
\pause
\ii
We'll get an infinite loop, since (* n (fact (- n 1))) will be
evaluated every time, even if we have hit the base case. 
\ei

\end{frame}
\end{document}