\documentclass{article}
\usepackage[margin=1in]{geometry}
\usepackage{enumitem}
\usepackage{hyperref}
\usepackage{fancyvrb}
\usepackage{multicol}
\usepackage{color}
\usepackage{longtable}

\setlength{\parindent}{0pt}
\setlength{\parskip}{1ex}


\newcommand{\myline}{\end{description}\hrulefill\begin{description}}
\newcommand{\be}{\mbox{}\begin{itemize}}
\newcommand{\ee}{\end{itemize}}
\newcommand{\bop}
           {\href{http://www.people.vcu.edu/~rhammack/BookOfProof/}
             {{\em      Book    of Proof}}}
\newcommand{\spl}
           {\href{http://www.scheme.com/tspl3/}
             {{\em The Scheme Programming Language}}}
\newcommand{\sicp}           
           {\href{https://mitpress.mit.edu/sicp/}
             {{\em Structure and Interpretation of Computer Programs}}}
\newcommand{\theory}
           {\href{http://cg.scs.carleton.ca/~michiel/TheoryOfComputation/}
             {{\em The Theory of Computation}}}

\newcommand{\automata}
           {\href{http://users.utu.fi/jkari/automata/}
             {{\em Automata and Formal Languages Lecture Notes}}}
           
\begin{document}
\centerline{\Large\bf CSCI 301: Formal Languages and Functional Programming}
\centerline{\large\bf Syllabus, Winter, 2017}


\begin{description}


\item[Instructor:] Geoffrey Matthews\\
{\em Email:} {\tt geoffrey dot matthews at wwu dot edu}\\
{\em Office hours:} MTWF 10:00-10:50, CF 469


\item[Lectures:] \mbox{}\\
\begin{tabular}{ccc}
  CRN & Time & Room \\\hline
  11650, 11652 & MTWF 9:00-9:50 & CF 225 \\
  12762, 12763 & MTWF 11:00-11:50 & AH 014 
  \end{tabular}

\item[Lab sessions:]\mbox{}\\
  \hfill
  \begin{tabular}{cccc}
    CRN & Time & Room & TA\\\hline
  11650 & T 10:00-11:50 & CF 414 & Katie Hursch\\
  11652 & W 10:00-11:50 & CF 414 & Ellyn Ayton\\
  12762 & T 12:00-01:50 & CF 416 & Elizabeth Brooks \\
  12763 & W 12:00-01:50 & CF 416 & Elizabeth Brooks \\
  \end{tabular}
  \hfill
  \mbox{}
  
You may attend the session you did not sign up for on a space
available basis.  There are no labs the first and last weeks of class,
and the week of the midterm.

\item[Catalog copy:] Introduction to discrete structures important to
  computer science, including sets, trees, functions, and
  relations. Proof techniques. Introduction to the formal language
  classes and their machines, including regular languages and finite
  automata, context free languages and pushdown automata. Turing
  machines and computability will be introduced. Programming using a
  functional language is required in the implementation of
  concepts. Includes lab.

\item[Goals:]  On completion of this course, students will demonstrate:
\begin{enumerate}
\item
Thorough understanding of the mathematical definitions of concepts
important to computer science, including sets, tuples, lists, strings,
languages, graphs, trees, functions and relations
\item
The ability to prove basic theorems involving these
mathematical concepts
\item
The ability to employ effectively the functional programming style in a
functional programming language
\item
Solid understanding of fundamental classes of languages, including
regular and context free, and their corresponding machines
\item
Basic understanding of Turing machines and computability
\item
Basic understanding of important algorithms, including conversion of
finite automata to different forms, conversion of grammars to machines
\item
Basic understanding of LL(k) and LR(K) grammars and the parsing
techniques used for those grammars
\end{enumerate}

\item[Websites:]\mbox{}
\begin{itemize}
  \item For class materials: \url{https://github.com/geofmatthews/csci301}
  \item For turning in homework and
    grading: \url{https://wwu.instructure.com/} 
\end{itemize}


\item[Goals:] This class is an introduction to computer science {\em
    theory}.  This is exactly parallel to the distinction between
  theoretical physics and applied physics.  We will use simplified,
  abstract, ideal mathematical models of computers, so that we can
  study the theoretical limits of what they can accomplish.  We will
  begin studying the basic mathematics we need, and then study
  mathematical models of computers of increasing complexity and power.
  Two classes of computers, finite state automata and push-down
  automata, also form the basis of powerful programming paradigms
  useful in a variety of situations.

  We will also study the functional programming language {\em Scheme}.
  Its simplicity, power, and mathematical elegance will inform our
  study of computers in the abstract, and also teach us new styles of
  programming.

\item[Texts:] \mbox{}
  \begin{description}
    \item[Math:] \mbox{}
\begin{itemize}
\item \url{http://www.people.vcu.edu/~rhammack/BookOfProof/}
\item \url{http://cg.scs.carleton.ca/~michiel/TheoryOfComputation/}
\item  \url{http://users.utu.fi/jkari/automata/} Lecture notes.
\item Other readings and handouts as they come up.
\end{itemize}
\item[Scheme:] \mbox{}
  \begin{itemize}
  \item  \url{http://www.scheme.com/tspl3/}  Note: Use the 3rd edition
  and do {\em not} use the
  4th edition of this book.  Our software is based on R5RS scheme and
  the 3rd edition covers this.  The 4th edition is based on R6RS scheme.
  \item The {\em help desk} within Racket leads to documentation on
    the implementation.
  \item  Optional:  \url{http://mitpress.mit.edu/sicp/}
\item Optional: \url{http://ds26gte.github.io/tyscheme/} 
\item Other readings and handouts as they come up.
  \end{itemize}
  \end{description}

\item[Software:] {\em DrRacket}, available here: 
\url{http://racket-lang.org/}

\item[Exams:] A midterm and a final, according to the schedule below.
  The final will be cumulative.

  Exams are closed book, with the exception that you may consult two
  pieces of paper during the exam.  You may write or print whatever
  you wish on these pieces of paper.

\item[Laboratory exercises:]
Laboratory exercises  will be handed out every week.
Each lab is due by midnight Tuesday of the following week.

\item[Math work:] Math assignments will be handed out as they come up
  during the quarter.  These assignments must be formatted with the
  \LaTeX\ technical typesetting system and submitted online.

  Pedagogically, it is important that you do as much independent work
  as possible.  The assigned homework represents a minimum, but there
  are many exercises in the books and online to get more practice at
  the math involved.

  I will attempt to grade the homework in a timely fashion.  This may
  necessitate grading only a sample of the problems submitted.

  Points will be awarded to each assignment based on its difficulty.
  Note that there may be an assignment due during dead week.

\item[Assessment and grades:]

Your grade will be based on the math homework, labs, and the two
exams. Weights are as follows.  

Please note that Canvas sometimes adds up raw scores and gives you a
percentage.  This usually does not reflect your weighted score for the
class.

Grades will be assigned based on scores as shown.  At the discretion
of the instructor, scores may be scaled.  Awarding $\pm$ is at the
discretion of the instructor.

\begin{tabular}{|c|c|c|c|c|c|}\hline
Homework & Labs & Midterm & Final\\\hline
25\% & 25\%  & 20\% & 30\% \\\hline
\end{tabular}\hfill
\begin{tabular}{|c|c|c|c|c|c|}\hline
\% & 90-100 & 80-89 & 70-79 & 60-69 & 0-60\\\hline
Grade & A & B & C & D & F\\\hline
\end{tabular}

\item[Late work:] Submissions are due before midnight of the due
  date. Anything turned in up to midnight of the following day will be
  accepted with a 25\% penalty. Anything turned in up to midnight of
  the second day after the due date will be accepted with a 50\%
  penalty.  Anything more than 48 hours late will not be accepted.  It
  is your responsibility to make sure all assignments are correctly
  submitted.  If you have trouble submitting on Canvas, please contact
  your TA or instructor immediately.

\item[Lectures and Reading:] The amount of material covered in this
  class is impossible to absorb with only four hours of lecture a
  week.  It is essential that you go over the assigned readings at
  least once before class discussion of that material, and attempt to
  work some of the solved problems.

  Many times, class lectures will consist of interactively solving
  problems, with class participation.  Some of the concepts and ideas
  necessary for solving these problems may not have been discussed
  previously in lectures.  In order to participate actively in the
  classroom discussion, you will have to read ahead.  Assigned
  readings for each lecture are in the schedule, below.  Note that
  this schedule may change as the quarter progresses.

\item[Attendance policy:] Attendance is not required but strongly
  recommended.  Studies show that regular attendance is highly
  correlated with performance.  

  You are responsible for all material covered in the lectures, books,
  handouts, or other assigned reading.  If you miss a lecture, make
  sure you get notes from other students and talk it over with them.

  If you have an emergency for one of the exam days, notify me as soon
  as possible.  I will handle each case individually and may, for
  example, schedule a make-up exam, or adjust your remaining scores to
  determine your grade.

\item[Academic dishonesty:] Please read Appendix D of WWU's Catalog on
  Academic Dishonesty.  It is available online at
  \url{http://catalog.wwu.edu}.

  Unless specified otherwise, all work for this course is meant to
  be done {\bf individually.}  The work that you turn in for a grade
  must be completely your own, or you will be guilty of academic
  dishonesty and could receive an F for the course.

  However, it can be a valiable learning experience to discuss
  work with your fellow students, and this is encouraged.
  However, after working with a colleague, {\bf you may not keep any
    paper or electronic copies of anything you produced together!}
  You may only keep your memories.  In particular, this means that
  {\bf you may not ask for or give help while sitting in front of a
    computer where the assignment is open!}  Also, {\bf you may not
    use anything a colleague has emailed to you!}  Delete the email
  and do not save a copy.

  To help understand what I mean, remember the {\fbox{\bf Long Term
    Memory Rule}}.  You may discuss, sketch, write things down, use
  your computers, whatever, but after you are done working with your
  fellow students all files must be deleted, whiteboards erased, and
  all papers you created must be destroyed.  You should then watch a
  rerun of {\em the Simpson's}, play a game of ping-pong, take a walk,
  or something else for half an hour. After this you can go back to
  your assignment (alone) and use the knowledge you have now gained.

  It is very easy for experienced software developers like your
  instructor and your TA to detect copied assignments.  Please do not
  put us in a situation where we have to fail you for plagiarism.

\end{description}

  
{\bf Schedule:}

\begin{longtable}{ll}
Wed, Jan 4:
&   Introduction, explanation of syllabus, expectations of students.
\\&   How to learn.
\\&   \LaTeX
\\
Fri, Jan 6: 
&  Introduction to Scheme and the Racket environment.
\\& \spl, Chapters 1 and 2.
\\ &  {\bf Lab 1 assigned.}
\\\hline
Mon, Jan 9:
& Continue introduction to Scheme and Lab 1.
\\&  Sets. \bop, Chapter 1.
\\
Tue, Jan 10:
& Logic. \bop, Chapter  2.
\\
Wed, Jan 11:
&  Counting, \bop, Chapter 3.
\\
Fri, Jan 13:
& More Scheme. \sicp, Chapter 1.
\\&  {\bf Lab 2 assigned.} 
\\\hline
Mon, Jan 16:&  MLK day, no classes.
\\
Tue, Jan 17:
& {\bf Lab 1 due.}
\\&  Direct proof, \bop Chapter 4.
\\&   Contrapositive Proof, \bop, Chapter 5.
\\
Wed, Jan 18:
& Proof by contradiction, \bop, Chapter 6.
\\& Proving non-conditional statements. \bop, Chapter 7.
\\
Fri, Jan 20:
& More Scheme.   \sicp, Chapter 2.
\\&  Proofs involving sets, \bop, Chapter 8.
\\& Disproof. \bop, Chapter 9.
\\& {\bf Lab 3 assigned.}
\\\hline
Mon, Jan 23:
& Mathematical induction.  \bop, Chapter 10.
\\
Tue, Jan 24:
& {\bf Lab 2 due.}
\\&  Relations and Functions.   \bop, Chapters 11, 12.
\\
Wed, Jan 25:
& Cardinality of Sets.  \bop, Chapter 13.
\\
Fri, Jan 27:
& {\bf Lab 4 assigned.}
\\\hline
Mon, Jan 30:
& Deterministic finite automata.
\\&\theory, 2.1, 2.2.
\\&\automata,    2.1.
\\
Tue, Jan 31:
& {\bf Lab 3 due.}
\\& Nondeterministic finite automata.
\\&\theory, 2.4, 2.5.
\\&\automata, 2.2, 2.3.
\\
Wed, Feb 1:
& Regular expressions, closure properties.
\\&\theory, 2.7,   2.8.
\\&\automata, 2.4, 2.6.
\\&\url{http://cs.stackexchange.com/questions/2016/how-to-convert-finite-automata-to-regular-expressions}
\\
Fri, Feb 3: & Snow day.
\\\hline
No lab this week.
\\
Mon, Feb 6:
& Snow day. No classes.
\\
Tue, Feb 7:
& {\bf Lab 4 due.}
\\ & Snow day.
\\
Wed, Feb 8:
& The pumping lemma for regular languages.
\\&\theory, 2.9.
\\&\automata, 2.5.
\\
Fri, Feb 10:& The pumping lemma for regular languages.
\\& {\bf Lab 5 assigned.}
\\\hline
Mon, Feb 13:
& Context-free grammars.
\\&\theory, 3.1, 3.2, 3.3.
\\&\automata, 3.1,  3.2.
\\
Tue, Feb 14:
& Review for midterm.
\\
Wed, Feb 15:
& {\bf Midterm exam.}
\\
Fri, Feb 17:
& {\bf Lab 6 assigned.}
\\&Review midterm.
\\\hline
Mon, Feb 20:&  President's day, no classes.
\\
Tue, Feb 21:
& Chomsky normal form.  \\&\theory, 3.4. \\&\automata, 3.3.
\\
Fri, Feb 24:
& {\bf Lab 7 assigned.}
\\& Pushdown automata. \\&\theory, 3.5, 3.6.  \\&\automata, 3.4.
\\\hline
Mon, Feb 27:
& Equivalence of PDA and CFG.  LL parsing. \\&\theory, 3.7.
\\
Tue, Feb 28:
& {\bf Lab 6 due.}
\\& Pumping lemma for CFL.  \\&\theory, 3.8.  \\&\automata, 3.5.
\\
Wed, Mar 1:
& Closure properties for CFL. \\&\automata, 3.6.
\\& Decision algorithms for regular languages. \\&\automata, 2.7.
\\& Decision algorithms for context free languages. \\&\automata, 3.7.
\\
Fri, Mar 3:
\\& Turing machines.
\\& \theory, 4.1, 4.2, 4.3, 4.4.
\\& \automata, 4.1, 4.2, 4.3, 4.4.
\\\hline
No lab this week.
\\
Mon, Mar 6:
& Decidability
\\& \theory, 5.1, 5.2, 5.3.
\\& \automata, 4.5, 4.6, 4.7.
\\
Tue, Mar 7:
& {\bf Lab 7 due.}
\\& Enumerability
\\&\theory, 5.4, 5.5, 5.6, 5.7, 5.8.
\\
Wed, Mar 8:&  Review.
\\
Fri, Mar 10: & Review.
\\\hline
Tue, Mar 14: &  Final Exam, 11:00am lecture, 8:00-10:00am
\\
Thu, Mar 16: &  Final Exam, 9:00am lecture, 8:00-10:00am
\end{longtable}



\end{document}
