\documentclass{article}
\usepackage{amsfonts,amssymb}
\usepackage[margin=1in]{geometry}

\title{CSCI 480, Winter 2017\\Math Exercises \# 1}
\author{YOUR NAME HERE}
\date{Due date:  Monday, January 23, midnight.}

\begin{document}

\maketitle

\begin{description}
\item[Exercises for Section 1.1]
\item[C.] Find the following cardinalities
\item[30.] $|\{\{1,4\},a,b,\{\{3,4\}\},\{\emptyset\}\}|$

\item[Exercises for Section 1.3]
\item[A.] List all the subsets of the following sets.
\item[8.] $\{\{0,1\},\{0,1,\{2\}\},\{0\}\}$

\item[Exercises for Section 1.4]
\item[A.] Find the indicated sets.
\item[12.] $\{X\in\mathcal{P}(\{1,2,3\}):2\in X\}$
\item[B.] Suppose that $|A|=m$ and $|B|=n$.  Find the following cardinalities.
\item[18.] $|\mathcal{P}(A\times \mathcal{P}(B))|$


\item[Exercises for Section 2.5] Write a truth table for the logical
  statements in problems 1--9:

\item[8.] $P \vee (Q \wedge \neg R)$

\item[Exercises for Section 2.10]  Negate the following sentences.


\item[8.] If $x$ is a rational number and $x\not = 0$, then
  $\tan(x)$ is not a rational number.
  

\item[Exercises for Section 3.1]

\item[4.]  Five cards are dealt off a standard 52-card deck and
  lined up in a row.  How many such lineups are there in
  which all 5 cards are of the same suit?

\item[Exercises for Section 3.2]

\item[8.] Compute how many 7-digit numbers can be made from the digits
  1,2,3,4,5,6,7 if there is no repetition and the odd digits
  must appear in an unbroken sequence.  (Examples: 3571264 or 2413576 or
  2467531, but not 7234615.)

\item[Exercises for Section 3.3]

\item[12.] Twenty-one people are to be divided into two teams, the Red
  Team and the Blue Team.  There will be 10 people on the Red Team and
  11 people on the Blue Team.  In how many ways can this be done?

\item[Exercises for Section 3.5]


\item[8.] This problem concerns 4-card hands dealt off of a standard
  52-card deck.  How many 4-card hands are there for which all 4 cards
  are of different suits or all 4 cards are red?




\end{description}


\end{document}
